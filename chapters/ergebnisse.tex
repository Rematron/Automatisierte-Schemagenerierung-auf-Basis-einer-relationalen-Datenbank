\chapter{Ergebnisse}
\label{cha:Ergebnisse}

\section{Diskussion}
Mit dem fertigen Implementierung ist es nun möglich, XML-Dokumente aufgrund von bereits bestehenden Schnittstellendefintionen zu validieren. Dies bringt besonders einen Mehrwert bei der Einführung von neuen Feldern, da die grundsätzlichen Überprüfungen nicht mehr notwendig sind. 
Auch können die XML Schemata an Fremdsysteme weitergegeben, um eine Abstimmung vor einer Systemeinführung sicher zu stellen.
Mithilfe von Schematron ist es außerdem möglich, kontextabhängige Regeln zu definieren. Leider ist aber derzeit bei BMD noch eine veraltete Microsoft-Komponente zum XML-Validierung im Einsatz. Damit ist eine Validierung mit Schematron im System nicht automatisch realisierbar.


\section{Evaluation}
Hier kommt die Evaluation hin...