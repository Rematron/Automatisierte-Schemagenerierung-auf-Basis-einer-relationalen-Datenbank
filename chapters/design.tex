\chapter{Konzeption}
\label{cha:Konzeption}
Nun soll beschrieben werden, welche logischen Schritte notwendig sind, um aus Basis einer Datenmenge ein XML Schema zu Validierung generieren zu können.

\section{Analyse der Daten}
Jeder Datensatz der Definition gehört auf mehrere Aspekte untersucht.
Die wichtigsten davon sind:

\begin{enumerate}
\item \textbf{Namen}: Bestimmen welche Bezeichnung das Element erhält.
\item \textbf{Datentyp}: Kann der Datentyp über einen vordefinierten Typen beschrieben werden, oder muss ein eigener komplexer Datentyp definiert werden. 
\item \textbf{Rekursion}: Bestimmen ob das Element aus weiteren Kind-Elementen besteht. 
\end{enumerate}

\section{Beschreibung des Aufbaus}
Jedes valide XML Dokument besteht aus einem Mindestanzahl an vorgeschriebenen Beschreibung die definiert gehören.
Für eine Schema-Datei zur Validierung ist dabei notwendig:

\begin{enumerate}
\item \textbf{XML Header}: Definition der Version und des Encoding.
\item \textbf{Namensraum-Definition}: Definiert die verwendeten Namensräume im Element und den Namensraum für die definierten Typen in dem Schema.
\item \textbf{Basis-Element}: Das Hauptelement beinhaltet nur ein Element und definiert jeden einzelne Schnittstelle. Dies wurde von \BMD vorgeschrieben.
\item \textbf{Definition der Schnittstelle}: Beinhaltet die einzelnen definierten Typen und beschreibt ihren Aufbau. Diese können auch weitere Kind-Elemente definieren.
\end{enumerate}

\section{Plausibilität der Daten}
Nachdem der Aufbau der zu übermittelten XML-Dateien durch das XML-Schema definiert wurde, folgt am Ende die Beschreibung des Schematron zur Überprüfung der Plausibilität.
Dieses werden am Ende des Dokuments angefügt. Da die Regeln jedoch nicht in der Datenmenge hinterlegt sind, müssen diese für jede Schnittstelle manuell definiert werden.
