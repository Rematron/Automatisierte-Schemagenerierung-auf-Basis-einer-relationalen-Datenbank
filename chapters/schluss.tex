\chapter{Conclusio}
\label{cha:Schluss}
Zum Schluss werden die erreichten Ziele zusammengefasst und ein Ausblick gegeben, welche weiterführenden Schritte notwendig sind, um den Prototypen in einem System komplett einsetzen zu können.

\section{Zusammenfassung}
Im Zuge der Implementierung wurde der gesamte Aufbau der \emph{Computer-aided manufactoring} (CAM)-\-Schnitt\-stelle reflektiert. 
Um hier einen einheitlichen Aufbau zu ermöglichen, wurden die Felddefinitionen für Import und Export funktional sauber getrennt. 
Die XML-Schemata können deswegen getrennt voneinander generiert werden.
Die Implementierung des Generators wurde im internen Qualitätstest geprüft und durchlief schlussendlich positiv einen Abnahmetest des Supports.
Die Validierung der Schemata wurde nur für den Import realisiert, da es den größeren Nutzen für BMD bringt. 
Die Implementierung des Schematron ist für einen Anwendungsfall realisiert wurden. Somit konnte bewiesen werden, dass die Umsetzung korrekt funktioniert.



\section{Ausblick}
Nach der erfolgreichen Implementierung soll der Generator bei ersten Kunden genutzt werden. Dies wird im Zuge von neuen Kundeninstallationen realisiert werden, um möglichst wenig Supportaufwand zu produzieren.
Als nächstes Ziel ist die Umsetzung der Validierung im Export vorgesehen. Der Umsetzungstermin ist projektabhängig und kann noch nicht genau definiert werden.
Der Ausbau der Regeln für das Schematron wird nach Bedarf weiter durchgeführt werden. Die veraltete Microsoft-Komponente wird in Zukunft noch umgestellt werden, um eine automatische Validierung der Geschäftsregeln zu ermöglichen.


