\chapter{Schluss}
\label{cha:Schluss}

\section{Zusammenfassung}
Im Zuge der Implementierung wurde der gesamte Aufbau der CAM-\-Schnitt\-stelle reflektiert. 
Um hier einen einheitlichen Aufbau zu ermöglichen, wurde die Felddefinitionen für Import und Export funktional sauber getrennt. 
Die XML-Schema können deswegen getrennt voneinander generiert werden.
Die Funktion des Generator lief durch den internen Qualitätsablauf und durchlief schlussendlich positiv einen Abnahmetest des Supports.
Die Validierung der Schemata wurde nur für den Import realisiert, da es den größeren Nutzen für BMD bringt. 
Die Implementierung des Schematron ist nur für einen einzigen Anwendungsfall realisiert wurden. Dies wurde Aufgrund von Zeitmangel nicht weiter ausgebaut. 



\section{Ausblick}
Nach der erfolgreichen Implementierung soll der Generator nun bei ersten Kunden genutzt werden. Dies wird aber im Zuge von neuen Kundeninstallationen realisiert werden, um möglichst wenig Supportaufwand zu produzieren.
Als nächstes Ziel ist die Umsetzung der Validierung im Export vorgesehen. Der Umsetzungstermin ist aber projektabhängig und kann noch nicht genau definiert werden.
Der Ausbau der Regeln für das Schematron wird nach Bedürfnis weiter ausgebaut werden. Die veraltete Microsoft-Komponente wird in Zukunft noch umgestellt werden, um eine automatische Validierung der Geschäftsregeln zu ermöglichen.


