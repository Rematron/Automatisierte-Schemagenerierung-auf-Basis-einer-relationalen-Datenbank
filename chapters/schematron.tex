\chapter{XML Schematron}
\label{cha:Schematron}

\section{Vorteile}
Mit Schematron ist es möglich auch Geschäftslogik in der XML-Datei abzubilden.
Im Vergleich dazu war es ja im XML Schema nur möglich auf Datentypen, Bereichswerte oder z.B. auf Referenzelemente zu überprüfen.

\section{Funktionsweise}

\section{XPath}


\section{Validierung}
Da das Schematron ISO standardisiert ist, gibt es die Möglichkeit diese in bestehende Strukturen einzugliedern. Wie in \cite{Akhilesh_validatinga} dargestellt, ist es möglich über die XML Elemente \emph{annotation} und \emph{appinfo} die Schematron-Elemente direkt im XML Schema einzubinden.
