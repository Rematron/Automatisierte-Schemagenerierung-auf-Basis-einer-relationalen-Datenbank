\chapter{Abstract}

\begin{english} %switch to English language rules
This should be a 1-page (maximum) summary of your work in English.
%und hier geht dann das Abstract weiter...
\end{english}

Im englischen Abstract sollte inhaltlich das Gleiche
stehen wie in der deutschen Kurzfassung. Versuchen Sie daher, die
Kurzfassung prä\-zise umzusetzen, ohne aber dabei Wort für Wort zu
übersetzen. Beachten Sie bei der Übersetzung, dass gewisse
Redewendungen aus dem Deutschen im Englischen kein Pendant haben
oder völlig anders formuliert werden müssen und dass die
Satzstellung im Englischen sich (bekanntlich) vom Deutschen stark
unterscheidet (mehr dazu in Abschn.\ \ref{sec:englisch}). Es
empfiehlt sich übrigens -- auch bei höchstem Vertrauen in die
persönlichen Englischkenntnisse -- eine kundige Person für das
"`proof reading"' zu engagieren.

Die richtige Übersetzung für "`Diplomarbeit"' ist übrigens
schlicht \emph{thesis}, allenfalls  "`diploma thesis"' oder "`Master's thesis"', 
auf keinen Fall aber "`diploma work"' oder gar "`dissertation"'. 
Für "`Bachelorarbeit"' ist wohl "`Bachelor thesis"' die passende Übersetzung. 

Übrigens sollte für diesen Abschnitt die \emph{Spracheinstellung} in \latex\ von Deutsch
auf Englisch umgeschaltet werden, um die richtige Form der
Silbentrennung zu erhalten, die richtigen Anführungszeichen müssen allerdings selbst gesetzt werden %
(s.\ dazu die Abschnitte \ref{sec:sprachumschaltung} %
und \ref{sec:anfuehrungszeichen}).
