\chapter{Kurzfassung}

Bei der Einführung einer neuen Software in einem System gibt es oft den Wunsch bestehende Daten von einem fremden System zu übernehmen oder diesem System Daten zu liefern. Dies kann und wird sehr oft über den Austausch von XML-Dateien realisiert. Mit einem XML-Schema können solche Dateien überprüft und validiert werden.

Um nun nicht jede Änderung einer Schnittstelle auch im Schema mit pflegen zu müssen, sollte dies möglichst automatisiert werden. In dieser Arbeit soll nun gezeigt werden, welche Schritte notwendig sind, um ein XML-Schema aus einer relationaler Datenbasis generieren zu können. 
Auch soll eine Möglichkeit zur erweiterten Validierung der XML-Dateien angeboten werden. 
In der Praxis wurde ein Prototyp bei \BMD implementiert, der dem Entwickler die Anpassung der Schnittstellen erleichtern soll.